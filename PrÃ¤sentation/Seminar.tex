\documentclass[10pt,a4paper]{beamer}
\usepackage[utf8]{inputenc}
\usepackage{amsmath}
\usepackage{amsfonts}
\usepackage{graphicx}
\usepackage{amssymb}
\usepackage{mathrsfs}
\usepackage[utf8]{inputenc}
\usetheme{Warsaw}  %% Themenwahl

%Hamiltonian simulation algorithms for near-term quantum hardware
\title[]{Hamiltonian simulation algorithms for near-term quantum hardware}
\author{Patrick Bettermann}
\date{\today}
\newcommand*\oldmacro{}%
\let\oldmacro\insertshorttitle%
\renewcommand*\insertshorttitle{%
  \oldmacro\hfill%
  \insertframenumber\,/\,\inserttotalframenumber}


\begin{document}
\maketitle
\setbeamertemplate{section in toc}[circle]
\frame{\tableofcontents[]}


\section{Introduction}
\begin{frame} %%Eine Folie
  \frametitle{Attosecond Physics} 
  \begin{Definition} %%Definition
    1 Attosecond (as) $= 1*10^{-18}$ s
  \end{Definition}
  \begin{itemize}
 	\item Shortest timescale available to us in experiments
 	\item Special analysis methods are required
	\end{itemize}
\end{frame}



            
\subsection{Pulses}
\begin{frame}
  \frametitle{Pulses} 
  \begin{Definition}
  	Complex Wavefunction $U(t)=|A(t)|exp(i[\omega_{0}t+\varphi(t)])$\\
  	 \footnotesize{$|A(t)|$ magnitude of the envelope, angular frequency $\omega_{0}$,  phase $\varphi(t)=arg[A(t)]$} \\
  	\begin{columns}
    	\begin{column}{0.35\textwidth}
      		$\varphi(t+t_{0})\approx \varphi_{0}+\varphi't+\frac{1}{2}\varphi''t^2$\\
      		$I(t)=I_{0}exp(-2t^2/\tau^2)$
   		\end{column}
    	\begin{column}{0.5\textwidth}
     		$A(t)=A_{0}exp(-t^2/\tau^2)exp(iat^2/\tau^2)$
    	\end{column}
	\end{columns}
  
\end{Definition}
\vspace{0.3in}
\begin{columns}
    \begin{column}{0.48\textwidth}
       \includegraphics[width=\textwidth,height=0.4\textheight,keepaspectratio]
            {figures/irpulse_neu2.png}
    \end{column}
    \begin{column}{0.48\textwidth}
       \includegraphics[width=\textwidth,height=0.4\textheight,keepaspectratio]
            {figures/pulsechirp.png}
    \end{column}
\end{columns}
\end{frame}



\subsection{Spectrogram}
\begin{frame}
  \frametitle{short-time Fourier Transform} 
  \begin{Definition}
  $\Phi(\nu,\tau)=\int P(t)G(t-\tau)exp(-i2\pi\nu t)dt$
  \end{Definition}
  \vspace{0.21in}
  \begin{columns}
  \begin{column}{0.5\textwidth}
  $Gate: G(t-\tau)$\\
  $Gated\,Pulse: P(t)G(t-\tau)$
        	\includegraphics[width=\textwidth,height=0.5\textheight,keepaspectratio]
            {figures/5signal.png}
    \end{column}
    \begin{column}{0.5\textwidth}
   	 Spectrogram: $|\Phi(\nu, \tau)|^2$ \\
   	 $2\sqrt{2}sin(4\pi t+cos(2\pi*0,25t))$:
    		\includegraphics[width=\textwidth,height=0.5\textheight,keepaspectratio]
            {figures/5spectrogram.pdf}
   \end{column}
  \end{columns}
 % \vspace{0.2in}
\end{frame}



\subsection{Attosecond Streaking Spectroscopy}
\begin{frame}
  \frametitle{Attosecond Streaking Spectroscopy}
  %\begin{itemize}
  %	\item Photoionization of a gas target by a XUV pulse
  %	\item Electrons leave the surface with a kinetic energy
  %	\item IR laser pulse influences the photoelectron momentum
  %	\item Final momentum is measured
  %\end{itemize}
  \begin{center}
           \includegraphics[width=\textwidth,height=0.75\textheight,keepaspectratio]
            {figures/experimental_setup.jpg}
  \end{center}
  %Source: Direct Measurement of Light Waves, E.Goulielmakis, 2004
  %\vspace{ 0.1in}
  \footnotesize Direct Measurement of Light Waves, E.Goulielmakis, 2004, Science
  %Attosecond spectroscopy in condensed matter, Cavalieri et al., 2007, Nature volume 449
\end{frame}

\subsection{Applying the FROG scheme}
\begin{frame}
  \frametitle{Applying FROG to Attosecond Streaking}
    Attosecond Streaking Formula:
  \begin{center} $\Phi(p,\tau)=\int_{-\infty}^{+\infty}E_{X}(t)d(p+A_{L}(t+\tau))e^{-i\phi(p,t+\tau)}e^{i(p^2)/2-\Omega_{X}+W)t}dt$  \\
   \vspace{0.2in} \begin{flushleft}
    Modified Spectrogram:
      \end{flushleft}
  	$\hat{S}(p,\tau)=\frac{|\Phi(t,\tau)|^2}{|d(p)|^2}\approx \bigl|\int_{-\infty}^{+\infty}E_{X}G(t+\tau)e^{\frac{i}{2}p^2t}dt\bigl|^2$
  \end{center}
  \vspace{0.3in} 
   FROG Spectrogram: 
  \begin{center}
           $\tilde{S}=\bigl|\int_{-\infty}^{+\infty}P(t)G(t+\tau)e^{i\omega t}dt\bigl|^2$  
  \end{center}
  \vspace{0.5in} 
  \footnotesize The accurate FROG characterization of attosecond pulses from streaking measurements, J.Gagnon et al., 2008, Appl.Phys. B 92, 25-32
\end{frame}


\subsection{LSGPA}
\begin{frame}
\begin{flushleft}
Initial Guess:
\end{flushleft}
 \frametitle{Least Squares Generalized Projections Algorithm}
 \begin{center}
 \includegraphics[width=\textwidth,height=0.99\textheight,keepaspectratio]
            {figures/lsgpa_0.png}
 \end{center}
 \vspace{0.3in} 
 \footnotesize The accurate FROG characterization of attosecond pulses from streaking measurements, J.Gagnon et al., 2008, Appl.Phys. B 92, 25-32
\end{frame}


\begin{frame}
\begin{flushleft}
LSGPA Loop:
\end{flushleft}
\begin{itemize}
\item apply alternating constraints in time-/freq. domain
\end{itemize}
 \frametitle{Least Squares Generalized Projections Algorithm}
 \begin{center}
 \includegraphics[width=\textwidth,height=0.99\textheight,keepaspectratio]
            {figures/lsgpa_1.png}
 \end{center}
 %\vspace{0.3in} 
 \footnotesize The accurate FROG characterization of attosecond pulses from streaking measurements, J.Gagnon et al., 2008, Appl.Phys. B 92, 25-32
\end{frame}

\section{Implementation}
\subsection{Problem Statement}
\begin{frame}
  \frametitle{Problem Statement}
  \begin{columns}
  \begin{column}{0.7\textwidth}
        \includegraphics[width=\textwidth,height=0.8\textheight,keepaspectratio]
            {figures/roi.pdf}
    \end{column}
    \begin{column}{0.3\textwidth}
    	\begin{itemize}
    		\item Identify Regions-Of-Interest
  			\item Setup the data structures
    		\item Satisfy the data constraints
    		\begin{itemize}
     			 \item{FFT}
   				 \item{LSGPA}
   			\end{itemize}
    	\end{itemize}
    \end{column}
  \end{columns}
\end{frame}

\begin{frame}
 \frametitle{Data Constraints}
 %\begin{itemize}
 %	\item number of FFT points: power of 2
 %	\item $T=\frac{2\pi}{E_{max}},\,\, E_{max}=2*photon\,energy$
 %	\item Time Pulse Axis: scaled by T (same number of points)
 %\end{itemize}
 \begin{center}
 	 \includegraphics[width=\textwidth,height=0.8\textheight,keepaspectratio]
            {figures/gatetrick.png}
 \end{center}
 \footnotesize The accurate FROG characterization of attosecond pulses from streaking measurements, J.Gagnon et al., 2008, Appl.Phys. B 92, 25-32
\end{frame}

\subsection{Data Structures}
\begin{frame}
  \frametitle{Data Structures}
  \begin{columns}
  \begin{column}{0.48\textwidth}
        \begin{itemize}
  			\item object-oriented
  			\item modularized
  		\end{itemize}
    \end{column}
    \begin{column}{0.48\textwidth}
    	\begin{itemize}
    		\item expandable
    		\item built on top of the Scan framework
    	\end{itemize}
    \end{column}
  \end{columns}
  \begin{center}
  		 \includegraphics[width=\textwidth,height=0.6\textheight,keepaspectratio]
            {figures/class-diagram.png}
  \end{center}
\end{frame}

\subsection{Example Code}
\begin{frame}
 \frametitle{Example Code}
  \begin{center}
  		 \includegraphics[width=\textwidth,height=0.6\textheight,keepaspectratio]
            {figures/example_code.png}
  \end{center}
\end{frame}

\subsection{Functions}
\begin{frame}
	\frametitle{Setup Function}
	\includegraphics[width=\textwidth,height=0.1\textheight,keepaspectratio]
            {figures/setup_code.png}
     \begin{center}
    	 \includegraphics[width=\textwidth,height=0.9\textheight,keepaspectratio]
            {figures/setup1.png}
     \end{center}
\end{frame}


\begin{frame}
	\frametitle{Run the Algorithm}
	\includegraphics[width=\textwidth,height=0.1\textheight,keepaspectratio]
            {figures/run_code.png}
	\begin{center}
		\includegraphics[width=\textwidth,height=0.99\textheight,keepaspectratio]
            {figures/lsgpa_1.png}
	\end{center}
\end{frame}


\section{Results}
\subsection{Synthesized Data}
\begin{frame}
	\frametitle{Results: synthesized data}
	\includegraphics[width=\textwidth,height=0.6\textheight,keepaspectratio]
            {figures/working/results_synthesized_2.png}
	\linebreak   
	\begin{flushright}
		Merit = $\sqrt{\sum_{i=1}^{N_{\epsilon}}|S_{1,i}-IFFT(\tilde{S}_{2,i})|^2}$
	\end{flushright}	 
\end{frame}

\subsection{Measured Data}
\begin{frame}{Outline}
	\frametitle{Results: measured data}
	\includegraphics[width=\textwidth,height=0.6\textheight,keepaspectratio]
            {figures/working/result_measured_new.png}
    \begin{flushright}
		Merit =  $\sqrt{\sum_{i=1}^{N_{\epsilon}}|S_{1,i}-IFFT(\tilde{S}_{2,i})|^2}$
	\end{flushright}
\end{frame}

\section{Outlook}
\begin{frame}
\frametitle{Outlook}
\begin{itemize}
	\item Make the algorithm run perfectly, fix the bug
	\item Research Interest: get the delays between two traces
\end{itemize}
\end{frame}

\section{End}
\begin{frame}
\title{End}
	\begin{center}
		Thank you for your attention!
	\end{center}
\end{frame}

\section{Appendix}
\subsection{EM spectrum}
\begin{frame}
  \frametitle{EM spectrum} 
  \begin{center}
            \includegraphics[width=\textwidth,height=0.8\textheight,keepaspectratio]
            {figures/em.jpg}
  \end{center}
  source: Melissa Petruzzello et all., Encyclopædia Britannica Inc., 2017
\end{frame}


\subsection{Attosecond Formula}
\begin{frame}
  \frametitle{Attosecond Streaking Formula}
  \begin{center}
            $\Phi(p,\tau)=\int_{-\infty}^{+\infty}E_{X}(t)d(p+A_{L}(t+\tau))e^{-i\phi(p,t+\tau)}e^{i(p^2)/2-\Omega_{X}+W)t}dt$            
  \end{center}
  \vspace{0.3in} 
  \begin{center}
  	$\varphi(p,t)=\int_{t}^{\infty}(pA_{L}(t')+\frac{1}{2}A_{L}^2(t'))dt'$
  \end{center} 
   \vspace{1in} 
  \footnotesize The accurate FROG characterization of attosecond pulses from streaking measurements, J.Gagnon et al., 2008, Appl.Phys. B 92, 25-32
\end{frame}

\end{document}