\documentclass[10pt,a4paper]{beamer}
\usepackage[utf8]{inputenc}
\usepackage{amsmath}
\usepackage{amsfonts}
\usepackage{graphicx}
\usepackage{animate}
\usepackage{physics}
\usepackage{amssymb}
\usepackage{mathrsfs}
\usepackage[utf8]{inputenc}
\usetheme{Warsaw}  %% Themenwahl

%Hamiltonian simulation algorithms for near-term quantum hardware
\title[]{Hamiltonian simulation algorithms for near-term quantum hardware}
\author{Patrick Bettermann}
\date{\today}
\newcommand*\oldmacro{}%
\let\oldmacro\insertshorttitle%
\renewcommand*\insertshorttitle{%
  \oldmacro\hfill%
  \insertframenumber\,/\,\inserttotalframenumber}


\begin{document}
\maketitle
\setbeamertemplate{section in toc}[circle]
\frame{\tableofcontents[]}


\section{Hamiltonian simulation}

\begin{frame} %%Eine Folie
  \frametitle{Problem statement} 
  \begin{Definition} 
  Hamiltonian simulation: \\
  	"Given a description of a Hamiltonian H, and evolution time t, some initial state $\ket{\psi(0)}$  produce the final state $\ket{\psi(t)}$ (to some error $\epsilon$)"
  \end{Definition}
  \begin{itemize}
 	\item "The Hamiltonian of a system is the sum of the kinetic energies of all the particles, plus the potential energy of the particles associated with the system
	\end{itemize}
\end{frame}

            
\subsection{}
\begin{frame}
  \frametitle{Why is this a difficult problem?} 
  	\begin{Definition}
  	We assume that the quantum state is loaded into memory
	\end{Definition}
\vspace{0.3in}
 \begin{itemize}
 	\item a classical computer can't store the state efficiently
 	\item a classical computer cannot produce a complete description of the state
 	\end{itemize}
\end{frame}


\subsection{Schrödinger equation}
\begin{frame}
  \frametitle{Schrödinger equation} 
  \begin{Definition}
  $ H |\psi(t) = i \hbar \frac{\delta}{\delta t} \ket{\psi(t)} $
  \end{Definition}
  \vspace{0.21in}
  integrate both sides:\\
  \vspace{0.11in}
  \quad $ \ket{\psi(t)} = e^{-iHt/\hbar} \ket{\psi(0)}$\\
  \vspace{0.11in}
  break H down into potential and kinetic energy: \\
  \quad $E_{kin} = \frac{p^2}{2m}$ \\
  \vspace{0.11in}
  $e^{\frac{-ip^2t}{2m\hbar}}$ and $e^{\frac{-iE_{pot}t}{\hbar}}$ don't commute, so $ e^{-iHt/\hbar} = e^{\frac{-ip^2t}{2m\hbar}}e^{\frac{-iE_{pot}t}{\hbar}}$ doesn't hold
\end{frame}



\subsection{Lie-Trotter product formula}
\begin{frame}
  \frametitle{Lie-Trotter product formula}
  \begin{Definition}
  $ e^{A+B} = \lim_{n \to \inf}(e^{A/n}e^{B/n})^{n}  $
  \end{Definition}
  \begin{itemize}
  	\item product formula: simulate the sum-terms of a Hamiltonian by simulating each one separetly for a small time slice
  	\item $ H = A + B + C $
  	\item $ U = e ^ {-i(A+B+C)t} = (e^{-iCt/r}e^{-iBt/r}e^{-iCt/r})^{r} $
  	\item switching between kinetic and potential energy terms
  \end{itemize}
  we arrive at: $e^{\frac{-i\hat{H}t}{\hbar}} =  \lim_{N\to\infty} (e^{\frac{-ip^2t}{2m \hbar N}}e^{\frac{-\hat{V}(\hat{x})t}{\hbar N}})^N$
\end{frame}


\section{Implementation}
\subsection{Split operator}
\begin{frame}

\begin{Definition}
 Split operator
\end{Definition}
  
Algorithm: \\

\begin{enumerate}
\item Apply a half step of the potential propagator to $\psi(0)$ \\
\item Apply the Fourier transform: momentum basis \\
\item Apply a full step of the kinetic propagator on the momentum basis\\
\item Apply the Inverse Fourier transform: back to coordinate basis\\
\item Apply the second half step of the potential propagator\\
\end{enumerate}
\vspace{0.1in}
this algorithm results from splitting the propagator, substituting into the Schrödinger equation and projecting onto a coordinate basis $\ket{x}$
\end{frame}

\subsection{Classical Approach}
\begin{frame}
  \frametitle{Simulating a wavefunction}
  \begin{itemize}
  	\item blue line:  $\bra{\Psi}\ket{\Psi}$ 
  	\item orange line: potential energy, $\hat{V}$
  \end{itemize}
  \animategraphics[loop,autoplay, width=1\linewidth]{12}{figures/without_well/animation_without_well-}{0}{399}
\end{frame}

\begin{frame}
  \frametitle{Simulating a wavefunction}
  \begin{itemize}
  	\item blue line:  $\bra{\Psi}\ket{\Psi}$ 
  	\item orange line: potential energy, $\hat{V}$
  \end{itemize}
  \animategraphics[loop,autoplay, width=1\linewidth]{12}{figures/with_well/animation_with_well-}{0}{399}
\end{frame}




\section{Hamiltonian simulation algorithms for near-term quantum hardware}

\begin{frame}
  \frametitle{}
  
\end{frame}

\subsection{Example Code}
\begin{frame}
 \frametitle{Example Code}
  \begin{center}
  		 \includegraphics[width=\textwidth,height=0.6\textheight,keepaspectratio]
            {figures/example_code.png}
  \end{center}
\end{frame}

\subsection{Functions}
\begin{frame}
	\frametitle{Setup Function}
	\includegraphics[width=\textwidth,height=0.1\textheight,keepaspectratio]
            {figures/setup_code.png}
     \begin{center}
    	 \includegraphics[width=\textwidth,height=0.9\textheight,keepaspectratio]
            {figures/setup1.png}
     \end{center}
\end{frame}


\begin{frame}
	\frametitle{Run the Algorithm}
	\includegraphics[width=\textwidth,height=0.1\textheight,keepaspectratio]
            {figures/run_code.png}
	\begin{center}
		\includegraphics[width=\textwidth,height=0.99\textheight,keepaspectratio]
            {figures/lsgpa_1.png}
	\end{center}
\end{frame}


\section{Results}
\subsection{Synthesized Data}
\begin{frame}
	\frametitle{Results: synthesized data}
	\includegraphics[width=\textwidth,height=0.6\textheight,keepaspectratio]
            {figures/working/results_synthesized_2.png}
	\linebreak   
	\begin{flushright}
		Merit = $\sqrt{\sum_{i=1}^{N_{\epsilon}}|S_{1,i}-IFFT(\tilde{S}_{2,i})|^2}$
	\end{flushright}	 
\end{frame}

\subsection{Measured Data}
\begin{frame}{Outline}
	\frametitle{Results: measured data}
	\includegraphics[width=\textwidth,height=0.6\textheight,keepaspectratio]
            {figures/working/result_measured_new.png}
    \begin{flushright}
		Merit =  $\sqrt{\sum_{i=1}^{N_{\epsilon}}|S_{1,i}-IFFT(\tilde{S}_{2,i})|^2}$
	\end{flushright}
\end{frame}

\section{Outlook}
\begin{frame}
\frametitle{Outlook}
\begin{itemize}
	\item Make the algorithm run perfectly, fix the bug
	\item Research Interest: get the delays between two traces
\end{itemize}
\end{frame}

\section{End}
\begin{frame}
\title{End}
	\begin{center}
		Thank you for your attention!
	\end{center}
\end{frame}


\end{document}